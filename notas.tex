\documentclass[a4paper,10pt]{article}
\usepackage[utf8]{inputenc}

%opening
\title{}
\author{}

\begin{document}


\section{Encontros}

* Preferem horários fixos. 
* Enquete no moodle sobre horários.

\section{Questões}
\begin{eqnarray*}
 \vec{u} &=& u_1\vec{i} + u_2\vec{j} + u_3\vec{k}\\
 \vec{v} &=& v_1\vec{i} + v_2\vec{j} + v_3\vec{k}
\end{eqnarray*}

Como $\vec{u}$ e $\vec{v}$ estão no plano XY, $u_3=v_3=0$. 

Como $\|\vec{v}\|=0$, $\vec{v}=\vec{0}$.

Já $\vec{u}$ é unitario, então $u_1^2+u_2^2=1$.
$$u_1=\cos(\theta),~~u_2=\sin(\theta).$$

Obs.: Quando $\theta=0$, $\vec{u}=\vec{i}$.
Quando $\theta=\pi/2$, $\vec{u}=\vec{j}$,

\section{Produto misto}
\begin{eqnarray*}
 \vec{u}\times  \vec{v} \cdot \vec{w}=(\vec{u}\times  \vec{v}) \cdot \vec{w}=\underbrace{\vec{u}\times(  \vec{v} \cdot \vec{w})}_{ERRADO}
\end{eqnarray*}


\begin{eqnarray*}
\vec{u}&=&\vec{i}+2\vec{j}+3\vec{k}\\
\vec{v}&=&\vec{i}       -\vec{k}\\
\vec{w}&=&\vec{i}+2\vec{j}+\vec{k}\\
\end{eqnarray*}

\begin{eqnarray*}
\left|\begin{array}{ccc}
1&2&3\\       
1&0&-1\\
1&2&1
      \end{array}
\right| = 1\cdot 0 \cdot 1 + 2\cdot(-1)\cdot1 + 3\cdot 1\cdot 2 - 3\cdot 0\cdot 1 - 2\cdot 1\cdot 1 - 1 \cdot (-1) \cdot 2
\end{eqnarray*}



\end{document}
