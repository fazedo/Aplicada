\documentclass[a4paper,10pt]{book}
\usepackage[utf8]{inputenc}
\usepackage{graphicx}
\usepackage{amsmath}
\usepackage{amsfonts}
\usepackage{url}

\newcommand{\sen}{\operatorname{sen}\,}
\newcommand{\senh}{\operatorname{senh}\,}
\renewcommand{\sin}{\operatorname{sen}\,}
\renewcommand{\sinh}{\operatorname{senh}\,}


\begin{document}

\chapter{16 de outubro}
\section{Funções periódicas}

Uma função $f:\mathbb{R}\to \mathbb{R}$ (ou $f:\mathbb{R}\to \mathbb{C}$ ) é dita periódica se existe $T>0$ tal que:
$$f(t+T)=f(t),~~ \forall t\in \mathbb{R}.$$

Aqui $T$ é chamado período e a função é dita $T$-periódica.

{\bf Exemplos:} Você conhece? $\cos(t), \sin(t)$.

% 
{\bf Observação:} Se $f(t)$ é $T$-periódica, $\alpha f(t)$ também é.

{\bf Observação:} Se $T$ é período de uma função $f(t)$ então todo múltiplo inteiro de $T$ também é período:
 $$f(t+nT)=f(t).$$
% 

{\bf Perguntas:} A soma de funções periódicas é periódica?

{\bf Resposta:} Se $f(t)$ tem período $T_1$ e g(t) tem período $T_2$ e $\frac{T_1}{T_2}=\frac{n}{m}$, isto é, $mT_1 = nT_2$, então $f(t)+g(t)$ é periódica. Aqui $n$ e $m$ são inteiros positivos.
{\bf Dem:} Construtiva com $T=mT_1$:
$$f(t+T)+g(t+T)=f(t+nT_1)+g(t+mT_2)=f(t)+g(t)$$
% 

 {\bf Definição:} Se $f(t)$ é periódica e existe $T_f$ o menor período, então $T_f$ é chamado de período fundamental e $w_f:=\frac{2\pi}{T_f}$ é a frequência angular fundamental.

{\bf Pergunta:} Existem funções periódicas sem período fundamental?
$$f(t)=1$$
% 
\subsection{Polinômios trigonométricos e séries trigonométricas}
 Seja $T>0$, definimos polinômio trigonomético de grau $N$ uma função do tipo:
 \begin{equation}f(t)=\frac{a_0}{2}+ \sum_{n=1}^N \left[a_n \cos(w_n t) + b_n \sen(w_nt)\right] \end{equation}
 onde $w_n=\frac{2\pi n}{T}$.
% 

  Seja $T>0$, definimos série trigonométrica toda função do tipo:
\begin{equation}f(t)=\frac{a_0}{2}+ \sum_{n=1}^\infty \left[a_n \cos(w_n t) + b_n \sen(w_n t)\right] \end{equation}
 onde $w_n=\frac{2\pi n}{T}$.


 {\bf Exemplo:}
  Mostre que $T$ é um período para séries e  polinômios trigonométricos acima definidos.

 \subsection{Relações de ortogonalidade}
 \begin{eqnarray*}
 \int_{0}^T \sen\left(\frac{2\pi nt}{T}\right) \sen\left(\frac{2\pi mt}{T}\right)dt&=&\left\{
 \begin{array}{ll}0, &n\neq m\\ \frac{T}{2}, &n=m\neq 0 \end{array} \right.\label{rel_ort_ss}\\
 \int_{0}^T \cos\left(\frac{2\pi nt}{T}\right) \cos\left(\frac{2\pi mt}{T}\right)dt&=&\left\{
 \begin{array}{ll}0, &n\neq m\\ \frac{T}{2}, &n=m\neq 0\\ T,&n=m=0 \end{array} \right.\label{rel_ort_cc}\\
 \int_{0}^T \cos\left(\frac{2\pi nt}{T}\right) \sen\left(\frac{2\pi mt}{T}\right)dt&=&0\label{rel_ort_sc}\label{rel_ort_cs}
 \end{eqnarray*}

\subsection{Forma harmônica}
$$A\cos(wt-\phi)=A\left[\cos(wt)\cos(\phi)+\sin(wt)\sin(\phi)\right]$$

$$A\cos(wt-\phi)=A\cos(wt)\cos(\phi)+A\sin(wt)\sin(\phi)$$

$$A_n\cos(w_nt-\phi_n)=\underbrace{A\cos(\phi_n)}_{a_n}\cos(w_nt)+\underbrace{A\sin(\phi_n)}_{b_n}\sin(w_nt)$$

$$f(t)=A_0+\sum_{n=1}^\infty A_n\cos(w_nt-\phi_n)$$

{\bf Obs:} Para qualquer par $a_n$ e $b_n$, existem $A_n$ e $\phi_n$ que satisfazem a identidade e $A_n>0,~~\forall n>0$.

\subsection{Forma complexa ou exponencial}
\section{Forma exponencial}
A forma exponencial de uma série de Fourier é obtida quando se substiuem as funções trigonométricas $\sen(w_nt)$ e $\cos(w_nt)$ por suas representações em termos de exponenciais complexos, isto é
\begin{equation}\cos(w_nt)=\frac{e^{iw_nt}+ e^{-iw_nt}}{2}~~~\hbox{e}~~~\sen(w_nt)=\frac{e^{iw_nt}- e^{-iw_nt}}{2i}\end{equation}
\begin{eqnarray*}
f(t)&=&\frac{a_0}{2}+\sum_{n=1}^\infty a_n\cos(w_n t)+\sum_{n=1}^\infty b_n\sen(w_n t)\\
&=&\frac{a_0}{2}+\sum_{n=1}^\infty a_n\left(\frac{e^{iw_nt}+ e^{-iw_nt}}{2}\right)+\sum_{n=1}^\infty b_n\left(\frac{e^{iw_nt}- e^{-iw_nt}}{2i}\right)\\
\end{eqnarray*}
Reagrupando os termos e usando o fato que $\frac{1}{i}=-i$, temos:
\begin{eqnarray}\label{form_exp_1}
f(t)=\frac{a_0}{2}+\sum_{n=1}^\infty \frac{a_n-ib_n}{2}e^{iw_nt}+\sum_{n=1}^\infty \frac{a_n+ib_n}{2}e^{-iw_nt}
\end{eqnarray}
Agora observamos que as definições \ref{coef} dadas por  
\begin{eqnarray*}
   a_0&=& \frac{2}{T}\int_0^T f(t)dt = \frac{2}{T}\int_{-T/2}^{T/2} f(t)dt\\
   a_n&=& \frac{2}{T}\int_0^T f(t)\cos(w_n t)dt = \frac{2}{T}\int_{-T/2}^{T/2} f(t)\cos(w_nt)dt\\
   b_n&=& \frac{2}{T}\int_0^T f(t)\sen(w_n t)dt = \frac{2}{T}\int_{-T/2}^{T/2} f(t)\sen(w_nt)dt
  \end{eqnarray*}
Embora estas expressões estejam definadas apenas para $n>0$, elas fazem sentidos para qualquer $n$ inteiro. Neste caso, valem as seguintes identidades:
\begin{equation}a_{-n}=a_n,~~b_{-n}=-b_{n}~~b_0=0.\end{equation}
onde se usou que $w_{-n}=\frac{2\pi (-n)}{T}=-\frac{2\pi n}{T}=-w_n$ e a paridade das funções cosseno e seno.
Estendendo estas definições para qualquer inteiro, introduzimos os coeficientes $C_n$ dados por:
\begin{equation}\label{def_cn}
C_n = \frac{a_n - ib_n}{2}
\end{equation}
Observe que estes coeficientes estão definidos para para número inteiro $n$, assim temos:
\begin{equation}
C_0 = \frac{a_0 - ib_0}{2}=\frac{a_0}{2}
\end{equation}
e
\begin{equation}
C_{-n} = \frac{a_{-n} - ib_{-n}}{2}=\frac{a_{n} + ib_{n}}{2}
\end{equation}
Substituindo estas expressões para $C_0$, $C_{n}$ e $C_{-n}$ em (\ref{form_exp_1}), obtemos:
\begin{eqnarray*}
f(t)=C_0+\sum_{n=1}^\infty C_n e^{iw_nt}+\sum_{n=1}^\infty C_{-n}e^{-iw_nt}
\end{eqnarray*}
Escrevemos agora esta última expressão em um único somatório:
 \begin{eqnarray}\label{forma_exp}
 f(t)=\sum_{n=-\infty}^\infty C_n e^{iw_nt}
 \end{eqnarray}
 onde se usou que $w_{-n}=\frac{2\pi (-n)}{T}=-\frac{2\pi n}{T}=-w_n$
 Observamos também que os coeficientes $C_n$ podem ser escritos das seguinte forma mais enxuta:
 \begin{eqnarray*}
 C_n &=& \frac{a_n - ib_n}{2}\\
 &=& \frac{1}{T}\int_0^Tf(t)\left[\cos(w_n t)-i\sen(w_n t)\right]dt\\
 &=&\frac{1}{T}\int_0^Tf(t)e^{-iw_nt}dt =\frac{1}{T}\int_{-T/2}^{T/2}f(t)e^{-iw_nt}dt 
 \end{eqnarray*}


\subsection{Convegência}
(Teorema de Dirichlet( Seja $f$ uma função periódica de período $T$, suave por partes e descontínua no máximo em um número finito de saltos dentro de cada intervalo, então a série de Fourier converge em cada ponto $t$ para
\begin{equation}
\frac{f(t+)+f(t-)}{2},
\end{equation}
onde $f(t+)$ e $f(t-)$ são os limites laterais à direita e à esquerda, respectivamente. Observe que nos pontos $t$ onde $f(t)$ é contínua, então $f(t+)=f(t-)$ e a série de Fourier converge para $f(t)$.

 
\subsection{Interlúdio - Caculando coeficientes}

{Obs:}

Se $f(t)$ é $T$-periódica, então:
$$\int_0^T f(t)dt = \int_{a}^{T+a}f(t)dt,~~~\forall a \in \mathbb{R}$$

Considere a fução periódica da por:

 Seja $f(t)$ uma função dada por
 \begin{eqnarray*}
 f(t)&=&|t|, \ \ -1\leq t<1\\
 f(t+2)&=&f(t),\ \ \forall t\in\mathbb{R}.
 \end{eqnarray*}
 Essa função é suave por partes e contínua em todos os pontos. Portanto se aplica o teorema:
 \begin{center}
 \includegraphics{figs/cap_series_figura_1}\end{center}
 Observamos que essa é uma função par, ou seja, $f(t)=f(-t)$. A fim de explorar essa simetria, utilizaremos as fórmulas (\ref{coef}) envolvendo integrais simétricas, isto é,
   \begin{eqnarray*}
    a_0&=& \frac{2}{T}\int_{-T/2}^{T/2} f(t)dt\\
    a_n&=&  \frac{2}{T}\int_{-T/2}^{T/2} f(t)\cos(w_nt)dt\\
    b_n&=&\frac{2}{T}\int_{-T/2}^{T/2} f(t)\sen(w_nt)dt
   \end{eqnarray*}
 onde $T=2$ e $w_n=\frac{2\pi n}{T}=\pi n$. Logo,
   \begin{equation*}
    a_0= \int_{-1}^{1} |t| dt=2\int_{0}^{1} t dt=2\left[\frac{t^2}{2}\right]_0^1=1\\
 	\end{equation*}
 	\begin{eqnarray*}
    a_n=  \int_{-1}^{1} |t|\cos(\pi n t)dt&=&2 \int_{0}^{1} t\cos(\pi n t)dt\\&=&2\left[\frac{t\sen(\pi n t)}{\pi n}\right]_0^1-2\int_0^1\frac{\sen(\pi n t)}{\pi n}dt\\
 	&=&2\left[\frac{t\sen(\pi n t)}{\pi n}+\frac{\cos(\pi n t)}{\pi^2 n^2}\right]_0^1=2\frac{(-1)^n-1}{\pi^2n^2}\\
 	 \end{eqnarray*}
 	\begin{eqnarray*}
    b_n&=&\int_{-1}^{1} |t|\sen(\pi n t)dt=0.
   \end{eqnarray*}
 onde se usou que $|t|$, $|t|\cos(\pi n t)$ são funções pares em $t$ e $|t|\sen(\pi n t)$ é ímpar em $t$. Assim, temos
 \begin{equation}
 f(t)=\frac{1}{2}-\frac{4}{\pi^2}\left(\cos(\pi t)+\frac{1}{3^2}\cos(3\pi t)+\frac{1}{5^2}\cos(5\pi t)+\cdots\right)
 \end{equation}
 Observe que, quando $t=0$, obtemos como subproduto da série de Fourier da $f(t)$ a soma da seguinte série numérica:
 \begin{equation}\label{serie_inv_impar}
 1+\frac{1}{3^2}+\frac{1}{5^2}+\cdots=\frac{\pi^2}{8}.
 \end{equation}
 A figura \ref{fig_conv_triangular} apresenta os gráficos da série que representa a função $f(t)$ com um termo, dois termos e três termos.
 \begin{figure}[!ht]
 \begin{center}
 \includegraphics{figs/cap_series_figura_2}\end{center}
 \caption{\label{fig_conv_triangular}Gráficos de $f_0(t)=\frac{1}{2}$ (azul), $f_1(t)=\frac{1}{2}-\frac{4}{\pi^2}\cos(\pi t)$ (verde) e $f_2(t)=\frac{1}{2}-\frac{4}{\pi^2}\left(\cos(\pi t)+\frac{1}{3^2}\cos(3\pi t)\right)$ (vermelho).}
 \end{figure}

 {\bf Onda quadrada}

 Seja $g(t)$ uma função dada por
 \begin{eqnarray*}
 g(t)&=&-1, \ \ -1< t<0\\
 g(t)&=&0, \ \ t=0\ \hbox{ou}\ t=1\\
 g(t)&=&1, \ \ 0< t<1\\
 g(t+2)&=&g(t),\ \ \forall t\in\mathbb{R}.
 \end{eqnarray*}
 Essa função é suave por partes e contínua em todos os pontos exceto por saltos nos inteiros, onde a função vale a média aritmética dos limites laterais. Portanto se aplica o teorema \ref{teo_Dirichlet}.
 \begin{figure}[!ht]
 \begin{center}
 \includegraphics{figs/cap_series_figura_3}\end{center}
 \end{figure}
 Observamos que essa é uma função ímpar, ou seja, $f(t)=-f(-t)$. Novamente, utilizaremos as fórmulas (\ref{coef}) envolvendo integrais simétricas:
   \begin{equation*}
    a_0= \int_{-1}^{1} g(t) dt=0\\
 	\end{equation*}
 	\begin{eqnarray*}
    a_n=  \int_{-1}^{1} g(t)\cos(\pi n t)dt=0\\
 	 \end{eqnarray*}
 	\begin{eqnarray*}
    b_n=\int_{-1}^{1} g(t)\sen(\pi n t)dt=2\int_{0}^{1} g(t)\sen(\pi n t)dt&=&2\int_{0}^{1} \sen(\pi n t)dt\\
 	&=&\frac{2}{\pi n}\left[-\cos(\pi n t)\right]_0^1=2\frac{1-(-1)^n}{\pi n}
   \end{eqnarray*}

    Logo,
 \begin{equation}
 g(t)=\frac{4}{\pi}\left(\sen(\pi t)+\frac{1}{3}\sen(3\pi t)+\frac{1}{5}\sen(5\pi t)+\cdots\right).
 \end{equation}
 A figura \ref{fig_conv_quadrangular} apresenta os gráficos da série que representa a função $g(t)$ com um termo, dois termos, três termos e quatro termos.
 \begin{figure}[!ht]
 \begin{center}
 \includegraphics{figs/cap_series_figura_4}\end{center}
 \caption{\label{fig_conv_quadrangular}Gráficos de $g_0(t)=\frac{4}{\pi}\sen(\pi t)$ (azul), $g_1(t)=\frac{4}{\pi}\left(\sen(\pi t)+\frac{1}{3}\sen(3\pi t)\right)$ (verde), $g_2(t)=g(t)=\frac{4}{\pi}\left(\sen(\pi t)+\frac{1}{3}\sen(3\pi t)+\frac{1}{5}\sen(5\pi t)\right)$ (vermelho) e $g_3(t)=g(t)=\frac{4}{\pi}\left(\sen(\pi t)+\frac{1}{3}\sen(3\pi t)+\frac{1}{5}\sen(5\pi t)+\frac{1}{7}\sen(7\pi t)\right)$ (preto).}
\end{figure}

{\bf Exemplo} Seja $h(t)$ uma função dada por
 \begin{eqnarray*}
 f(t)&=&t, \ \ 0<t<1\\
 f(t)&=&\frac{1}{2}, \ \ t=1\\
 f(t+1)&=&f(t),\ \ \forall t\in\mathbb{R}.
 \end{eqnarray*}
 Essa função é suave por partes e contínua exceto por salto nos inteiros onde $h(t)$ assume o valor médio dos limites laterais. Portanto se aplica o teorema \ref{teo_Dirichlet}.
 \begin{figure}[!ht]
 \begin{center}
 \includegraphics{figs/cap_series_figura_5}\end{center}
 \end{figure}
 Utilizaremos as fórmulas (\ref{coef}) envolvendo integrais no intervalo $[0,1]$, isto é,
   \begin{equation*}
    a_0= 2\int_{0}^{1} t dt=2\left[\frac{t^2}{2}\right]_0^1=1\\
 	\end{equation*}
 	\begin{eqnarray*}
    a_n=  2\int_{0}^{1} t\cos(2 \pi n t)dt=&=&2\left[\frac{t\sen(2 \pi n t)}{2 \pi n}\right]_0^1-2\int_0^1\frac{\sen(2 \pi n t)}{2 \pi n}dt\\
 	&=&2\left[\frac{t\sen(2 \pi n t)}{2 \pi n}+\frac{\cos(2 \pi n t)}{4\pi^2 n^2}\right]_0^1=0\\
 	 \end{eqnarray*}
 	\begin{eqnarray*}
    b_n=  2\int_{0}^{1} t\sen(2 \pi n t)dt=&=&2\left[-\frac{t\cos(2 \pi n t)}{2 \pi n}\right]_0^1+2\int_0^1\frac{\cos(2 \pi n t)}{2 \pi n}dt\\
 	&=&2\left[-\frac{t\cos(2 \pi n t)}{2 \pi n}+\frac{\sen(2 \pi n t)}{4\pi^2 n^2}\right]_0^1=-\frac{1}{\pi n}\\
 	 \end{eqnarray*}
 Logo,
 \begin{equation}
 h(t)=\frac{1}{2}-\frac{1}{\pi}\left(\sen(2 \pi t)+\frac{1}{2}\sen(4\pi t)+\frac{1}{3}\sen(6\pi t)+\cdots\right).
 \end{equation}

 
 {\bf Obs:} Os coeficiente $b_n$ da série de Fourier de uma função par são nulos bem como os coeficiente $a_n$ da série de Fourier de uma função ímpar também o são.

Demonstre a observação .




 \chapter{21 de outubro}
\subsection{Diagramas de espectro}


Diagramas espectro são representações gráficas dos coeficientes de Fourier $C_n$ associados a uma função periódica $f(t)$. Como os coeficientes $C_n$ são números complexos, é comum representá-los na forma de módulo e fase, isto é:
 \begin{equation}C_n = |C_n|e^{i\phi_n}.\end{equation}
 O ângulo de fase assim definido coincide com o conceito de argumento do número $C_n$.

 {\bf Exemplo:} A função 
 \begin{equation}f(t)=-1+2\cos(t)+4\sen(2t)\end{equation}
 é periódica com periodo fundamental $2\pi$ e pode ser escrita na forma exponencial da seguinte forma:
 \begin{eqnarray*}
 f(t)&=&-1+2\left(\frac{e^{it}+e^{-it}}{2}\right)+4\left(\frac{e^{2it}-e^{-2it}}{2i}\right)\\
 &=&2i e^{-2it} + e^{-it}-1+e^{it}- 2ie^{2it}
 \end{eqnarray*}
 Assim, identificamos cinco coeficientes não nulos:
 \begin{equation*}
 \begin{array}{lclcll}
  C_{-2}&=&2i=2e^{\frac{i\pi}{2}} &\Longrightarrow& |C_{-2}|=2, ~~ &\phi_{-2}=\frac{\pi}{2}\\
  C_{-1}&=&1 &\Longrightarrow& |C_{-1}|=1, ~~ &\phi_{-1}=0\\
  C_{0}&=&-1=1e^{\pi} &\Longrightarrow& |C_{0}|=1, ~~ &\phi_0=\pi\\
  C_{1}&=&1 &\Longrightarrow& |C_{1}|=1, ~~ &\phi_1=0\\
  C_{2}&=&-2i=2e^{\frac{-i\pi}{2}} &\Longrightarrow& |C_{2}|=2, ~~ &\phi_2=-\frac{\pi}{2}
 \end{array}
  \end{equation*}
 Os digramas de espectro de amplitude e fase são dados a seguir:
 \begin{figure}[!ht]
 \includegraphics{figs/cap_diagramas_espectro_figura_2}
 \includegraphics{figs/cap_diagramas_espectro_figura_3}
 \end{figure}

 
{\bf Exemplo:} As primeiras raias do diagrama de espectro da função a seguir: 
 \begin{equation}
 g(t)=\cdots+\frac{2i}{5\pi}e^{-5i\pi t}+\frac{2i}{3\pi}e^{-3i\pi t}+\frac{2i}{\pi}e^{-i\pi t}-\frac{2i}{\pi}e^{i\pi t}-\frac{2i}{3\pi}e^{3i\pi t}-\frac{2i}{5\pi}e^{5i\pi t}-\cdots,
 \end{equation}
 são dados na figura a seguir
 \begin{figure}[!ht] 
 \includegraphics{figs/cap_diagramas_espectro_figura_4}
 \includegraphics{figs/cap_diagramas_espectro_figura_5}
 \end{figure}


 {\bf Exemplos:}
 
 $$f(t)=\sum_{n=1}^\infty \frac{1}{n^2}\cos(nt)$$
 
 $$g(t)=\sum_{n=1}^\infty \frac{1}{n^2}\sin(nt)$$
 
 
 \section{Propriedades das Séries de Fourier}
 
Define-se a potência média de um função periódica $f(t)$ como
\begin{equation}\overline{P}_f=\frac{1}{T}\int_0^T |f(t)|^2dt
\end{equation}

 A potência média da função $f(t)=A\cos(wt)$ é dada por
 \begin{eqnarray*}
 \overline{P}_f&=&\frac{1}{T}\int_0^T |f(t)|^2dt\\
 &=&\frac{1}{T}\int_0^{T} A^2\cos\left(\frac{2\pi}{T} t\right)^2dt\\
 &=&\frac{A^2}{T}\int_0^{T} \left(\frac{\cos\left(\frac{4\pi}{T} t\right)+1}{2}  \right)dt\\
 &=&\frac{A^2}{2}
 \end{eqnarray*}
  onde se usou que $w=\frac{2\pi}{T}$ e identidade trigonométrica dada por:
 \begin{equation}\cos^2(x)=\left(\frac{e^{ix}+e^{-ix}}{2}\right)^2=\frac{e^{2ix}+2+e^{-2ix}}{4}=\frac{\cos(2x)+1}{2}.\end{equation}

{\bf Exemplo:} Seja $V(t)=A\cos(wt)$ uma fonte de tensão com frequência $w=60$\ \!\!Hz $=120\pi$\ \!\!rad/s ligado a um resistor de resistência $R\Omega$. A potência no resistor é
 \begin{equation}
 P(t)=\frac{V(t)^2}{R}
 \end{equation}
 e a potência média $P_m$ é
 \begin{equation}
 P_m=\frac{1}{T}\int_0^TP(t)dt=\frac{1}{T}\int_0^T\frac{V(t)^2}{R}dt,
 \end{equation}
 onde $T=\frac{1}{60}s$. Por outro lado, a potência média é calculada em termos da tensão média por
 \begin{equation}
 P_m=\frac{V_m^2}{R},
 \end{equation}
 ou seja,
 \begin{equation}{\label{valor_RMS}}
 V_m^2=\frac{1}{T}\int_0^T V(t)^2 dt.
 \end{equation}
 O exemplo  nos dá o valor da potência média do sinal $V(t)=A\cos(wt)$. Logo,
 \begin{equation}
 V_m=\frac{A}{\sqrt{2}}.
 \end{equation}
 Se $V_m=127V$, então a amplitude do sinal é aproximadamente $A\approx 180$.

 
{\bf Observação:}
Na expressão (\ref{valor_RMS}), $V_m$ também é chamado de valor RMS do sinal $v(t)$ (Root mean square):
 \begin{equation}
 V_{RMS}=\sqrt{\frac{1}{T}\int_0^T V(t)^2 dt}.
 \end{equation}

 {\bf Teorema de Parseval} Seja $f(t)$ uma função periódica representável por uma série de Fourier, então vale a seguinte identidade.
 \begin{equation}\label{teo_parseval} 
 \frac{1}{T}\int_0^T |f(t)|^2dt=\sum_{n=-\infty}^\infty |C_n|^2.
  \end{equation}

{\bf Dem:}
 \begin{eqnarray*}
  \frac{1}{T}\int_0^T |f(t)|^2dt&=&\frac{1}{T}\int_0^T f(t)\overline{f(t)}dt
 \end{eqnarray*}
  Como $\displaystyle f(t)=\sum_{n=-\infty}^\infty C_n e^{iw_n t}$, temos
  \begin{eqnarray*}
   \overline{f(t)}&=&\overline{\sum_{n=-\infty}^\infty C_n e^{iw_n t}}
   =\sum_{n=-\infty}^\infty \overline{C_n}~ \overline{e^{iw_n t}}
   =\sum_{n=-\infty}^\infty \overline{C_n} e^{-iw_n t}
  \end{eqnarray*}
 Substituindo esta expressão para $\overline{f(t)}$ na definição de potência média, temos:
 \begin{eqnarray*}
  \frac{1}{T}\int_0^T |f(t)|^2dt&=&\frac{1}{T}\int_0^T f(t)\overline{f(t)}dt=\frac{1}{T}\int_0^Tf(t)\left[\sum_{n=-\infty}^\infty \overline{C_n} e^{-iw_n t}\right] dt\\
  &=&\frac{1}{T}\sum_{n=-\infty}^\infty\left[\overline{C_n}\int_0^Tf(t)e^{-iw_nt}dt\right]
  \end{eqnarray*}
  Como $C_n=\frac{1}{T}\int_0^Tf(t)e^{-iw_nt}dt$, temos:
 \begin{eqnarray*}
  \frac{1}{T}\int_0^T |f(t)|^2dt&=&\sum_{n=-\infty}^\infty\overline{C_n}C_n = \sum_{n=-\infty}^\infty|C_n|^2
  \end{eqnarray*}







 
 {\bf Exemplo}
Seja $g(t)$ um função dada no exemplo \ref{ex_quadrada}, isto é,
 \begin{eqnarray*}
 g(t)&=&-1, \ \ -1< t<0\\
 g(t)&=&0, \ \ t=0\ \hbox{ou}\ t=1\\
 g(t)&=&1, \ \ 0< t<1\\
 g(t+2)&=&g(t),\ \ \forall t\in\mathbb{R}.
 \end{eqnarray*}
 \begin{figure}[!ht]
 \begin{center}
 \includegraphics{figs/cap_propriedades_series_figura_1}\end{center}
 \end{figure}
 Vimos no exemplo \ref{ex_quadrada} que sua expansão em série de Fourier é da forma:
 \begin{equation}
 g(t)=\frac{4}{\pi}\left(\sen(\pi t)+\frac{1}{3}\sen(3\pi t)+\frac{1}{5}\sen(5\pi t)+\cdots\right).
 \end{equation}
 Calcularemos agora a potência média desta função através de sua representação no tempo e depois em frequência:
 \begin{eqnarray*}
  \overline{P_f}=\frac{1}{T}\int_0^T |g(t)|^2dt=\frac{1}{2}\int_0^2 |g(t)|^2dt=\frac{1}{2}\int_0^2 1dt=1
 \end{eqnarray*}
 Alternativamente, temos pelo Teorema de Parseval:
 \begin{eqnarray*}
  \overline{P_f}=\sum_{n=-\infty}^\infty |C_n|^2=\sum_{n=-\infty}^\infty \left|\frac{a_n-ib_n}{2}\right|^2=\frac{1}{4}\sum_{n=-\infty}^\infty |b_n|^2
 \end{eqnarray*}
 Como $b_{-n}=b_n$, temos que $|b_{-n}|=|b_n|$ e ainda temos que $b_0=0$, portanto: 
 \begin{eqnarray*}
  \overline{P_f}=\frac{1}{2}\sum_{n=1}^\infty |b_n|^2 = \frac{1}{2}\left(\frac{4}{\pi}\right)^2\left(1 + \frac{1}{3^2}+ \frac{1}{5^2}+ \frac{1}{7^2}+\cdots\right)
 \end{eqnarray*}
 temos:
 \begin{eqnarray*}
  \overline{P_f}=\frac{1}{2}\left(\frac{4}{\pi}\right)^2\frac{\pi^2}{8}=1
 \end{eqnarray*}

 
 
 
 
 
 
 
 
 
 
 
 
 \section{Passagem do discreto para o contínuo}
 
 \chapter{23 de outubro}
 \section{Exemplos de transformadas de Fourier}
 
 
 
\subsection{Função tenda assimétrica}

 \begin{equation}f(t)=\left\{\begin{array}{cc}e^{at}&\hbox{se}\ t<0\\
 e^{-bt}&\hbox{se}\ t>0
 \end{array}\right.
 \end{equation}
 onde $a$ e $b$ são constantes positivas. A figura  mostra o gráfico de $f(t)$ para $a=1$ e $b=3$.
 \begin{figure}[!ht]
 \begin{center}
 \includegraphics{figs/cap_transformada_de_fourier_figura_7}\end{center}
 \caption{\label{fig_trans_1} Gráfico de $f(t)=e^{t}$, se $t<0$ ou $f(t)=e^{-3t}$ se $t>0$.  }
 \end{figure}
 A transformada de Fourier de $f(t)$ é calculada da seguinte forma:
 \begin{eqnarray*}
 F(w)=\mathcal{F}\{f(t)\}&=&\int_{-\infty}^\infty f(t) e^{-i w t}dt\\
 &=&\int_{-\infty}^0 e^{at} e^{-i w t}dt+\int_{0}^\infty e^{-bt} e^{-i w t}dt\\
 &=&\int_{-\infty}^0 e^{at} \left(\cos(wt)-i\sen(wt)\right)dt+\int_{0}^\infty e^{-bt} \left(\cos(wt)-i\sen(wt)\right)dt\\
 &=&\int_{0}^\infty e^{-at} \left(\cos(wt)+i\sen(wt)\right)dt+\int_{0}^\infty e^{-bt} \left(\cos(wt)-i\sen(wt)\right)dt\\
 &=&\frac{a}{a^2+w^2}+\frac{iw}{a^2+w^2}+\frac{b}{b^2+w^2}-\frac{iw}{b^2+w^2}\\
 &=&\frac{a}{a^2+w^2}+\frac{b}{b^2+w^2}+i\left(\frac{w}{a^2+w^2}-\frac{w}{b^2+w^2}\right)
 \end{eqnarray*}
 onde se usou os itens 1 e 2 da tabela de integrais definidas.

 {\bf O que acontece quando $a=b$?}
 % 
% 
 \subsection{Delta de Dirac}
 
 Calculamos a transformada de Fourier do delta de Dirac $\delta(t-a)$, $a\in\mathbb{R}$ da seguinte forma:
 \begin{eqnarray*}
 F(w)=\mathcal{F}\{\delta(t-a)\}&=&\int_{-\infty}^\infty \delta(t-a) e^{-i w t}dt\\
 &=&e^{-i w a}
 \end{eqnarray*}

 \chapter{23 de outubro}
\end{document}
