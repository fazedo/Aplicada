\documentclass[a4paper,10pt]{book}
\usepackage[utf8]{inputenc}
\usepackage{graphicx}
\usepackage{amsmath}
\usepackage{amsfonts}
\usepackage{url}

\newcommand{\sen}{\operatorname{sen}\,}
\newcommand{\senh}{\operatorname{senh}\,}
\renewcommand{\sin}{\operatorname{sen}\,}
\renewcommand{\sinh}{\operatorname{senh}\,}


\begin{document}

\chapter{16 de outubro}
\section{Funções periódicas}

Uma função $f:\mathbb{R}\to \mathbb{R}$ (ou $f:\mathbb{R}\to \mathbb{C}$ ) é dita periódica se existe $T>0$ tal que:
$$f(t+T)=f(t),~~ \forall t\in \mathbb{R}.$$

Aqui $T$ é chamado período e a função é dita $T$-periódica.

{\bf Exemplos:} Você conhece? $\cos(t), \sin(t)$.

% 
{\bf Observação:} Se $f(t)$ é $T$-periódica, $\alpha f(t)$ também é.

{\bf Observação:} Se $T$ é período de uma função $f(t)$ então todo múltiplo inteiro de $T$ também é período:
 $$f(t+nT)=f(t).$$
% 

{\bf Perguntas:} A soma de funções periódicas é periódica?

{\bf Resposta:} Se $f(t)$ tem período $T_1$ e g(t) tem período $T_2$ e $\frac{T_1}{T_2}=\frac{n}{m}$, isto é, $mT_1 = nT_2$, então $f(t)+g(t)$ é periódica. Aqui $n$ e $m$ são inteiros positivos.
{\bf Dem:} Construtiva com $T=mT_1$:
$$f(t+T)+g(t+T)=f(t+nT_1)+g(t+mT_2)=f(t)+g(t)$$
% 

 {\bf Definição:} Se $f(t)$ é periódica e existe $T_f$ o menor período, então $T_f$ é chamado de período fundamental e $w_f:=\frac{2\pi}{T_f}$ é a frequência angular fundamental.

{\bf Pergunta:} Existem funções periódicas sem período fundamental?
$$f(t)=1$$
% 
\subsection{Polinômios trigonométricos e séries trigonométricas}
 Seja $T>0$, definimos polinômio trigonomético de grau $N$ uma função do tipo:
 \begin{equation}f(t)=\frac{a_0}{2}+ \sum_{n=1}^N \left[a_n \cos(w_n t) + b_n \sen(w_nt)\right] \end{equation}
 onde $w_n=\frac{2\pi n}{T}$.
% 

  Seja $T>0$, definimos série trigonométrica toda função do tipo:
\begin{equation}f(t)=\frac{a_0}{2}+ \sum_{n=1}^\infty \left[a_n \cos(w_n t) + b_n \sen(w_n t)\right] \end{equation}
 onde $w_n=\frac{2\pi n}{T}$.


 {\bf Exemplo:}
  Mostre que $T$ é um período para séries e  polinômios trigonométricos acima definidos.

 \subsection{Relações de ortogonalidade}
 \begin{eqnarray*}
 \int_{0}^T \sen\left(\frac{2\pi nt}{T}\right) \sen\left(\frac{2\pi mt}{T}\right)dt&=&\left\{
 \begin{array}{ll}0, &n\neq m\\ \frac{T}{2}, &n=m\neq 0 \end{array} \right.\label{rel_ort_ss}\\
 \int_{0}^T \cos\left(\frac{2\pi nt}{T}\right) \cos\left(\frac{2\pi mt}{T}\right)dt&=&\left\{
 \begin{array}{ll}0, &n\neq m\\ \frac{T}{2}, &n=m\neq 0\\ T,&n=m=0 \end{array} \right.\label{rel_ort_cc}\\
 \int_{0}^T \cos\left(\frac{2\pi nt}{T}\right) \sen\left(\frac{2\pi mt}{T}\right)dt&=&0\label{rel_ort_sc}\label{rel_ort_cs}
 \end{eqnarray*}

\subsection{Forma harmônica}
$$A\cos(wt-\phi)=A\left[\cos(wt)\cos(\phi)+\sin(wt)\sin(\phi)\right]$$

$$A\cos(wt-\phi)=A\cos(wt)\cos(\phi)+A\sin(wt)\sin(\phi)$$

$$A_n\cos(w_nt-\phi_n)=\underbrace{A\cos(\phi_n)}_{a_n}\cos(w_nt)+\underbrace{A\sin(\phi_n)}_{b_n}\sin(w_nt)$$

$$f(t)=A_0+\sum_{n=1}^\infty A_n\cos(w_nt-\phi_n)$$

{\bf Obs:} Para qualquer par $a_n$ e $b_n$, existem $A_n$ e $\phi_n$ que satisfazem a identidade e $A_n>0,~~\forall n>0$.
\subsection{Forma complexa}
\subsection{Diagramas de espectro}

\end{document}
